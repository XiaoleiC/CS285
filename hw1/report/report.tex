\documentclass[11pt]{article}
\usepackage[margin=1in]{geometry}
\usepackage{amsmath}
\usepackage{booktabs}
\usepackage{graphicx}
\usepackage{float}
\usepackage{pgfplots}
\usepackage{caption}
\pgfplotsset{compat=1.18}

\title{CS285 Assignment 1 Report\\Imitation Learning for Push-T}
\author{XiaoLei Chu \\ SID: 3038525739}
\date{\today}

\begin{document}
\maketitle

\section{Overview}
This report summarizes my Homework 1 experiments for CS285 (Push-T imitation learning) using:
\begin{itemize}
    \item an MSE policy for action chunk prediction;
    \item a flow matching policy with Euler integration at inference time.
\end{itemize}
Both runs used the provided training and evaluation pipeline, and both include full WandB logs/videos/checkpoints in the submission package.

\section{Experimental Setup}
\textbf{Shared training configuration (both runs):}
\begin{itemize}
    \item seed: 42
    \item chunk size: 8
    \item batch size: 128
    \item optimizer: Adam, learning rate \(3\times 10^{-4}\), weight decay 0
    \item hidden dimensions: \((256, 256, 256, 256)\)
    \item epochs: 400
    \item eval interval: every 10,000 training steps
    \item flow denoising (Euler) steps: 10
\end{itemize}

\textbf{Curve data source:}
The train loss curves are generated from the exported WandB CSV file
\texttt{wandb\_export\_2026-02-11T18\_37\_24.546-08\_00.csv}. The eval reward curves are taken from each run's \texttt{log.csv}.

\textbf{MLP architecture used for MSE policy:}
\begin{itemize}
    \item input: normalized state (\(5\)-D);
    \item 4 fully connected hidden layers, each with 256 units;
    \item activation: ReLU after each hidden layer;
    \item output: flattened action chunk of size \(8 \times 2 = 16\), reshaped to \((8, 2)\).
\end{itemize}

\section{MSE Policy Results}
\subsection{Training Curves}
\begin{figure}[H]
\centering
\begin{tikzpicture}
\begin{axis}[
    width=0.48\textwidth,
    height=0.32\textwidth,
    xlabel={Training Step},
    ylabel={Train Loss},
    grid=both,
    title={MSE: Loss vs Step},
]
\addplot[color=blue, line width=1pt] table[x=step, y=loss, col sep=comma] {data/mse_loss.csv};
\end{axis}
\end{tikzpicture}
\hfill
\begin{tikzpicture}
\begin{axis}[
    width=0.48\textwidth,
    height=0.32\textwidth,
    xlabel={Training Step},
    ylabel={Eval Mean Reward},
    ymin=0,
    ymax=1,
    grid=both,
    title={MSE: Reward vs Step},
]
\addplot[color=red, mark=*, line width=1pt] table[x=step, y=reward, col sep=comma] {data/mse_reward.csv};
\end{axis}
\end{tikzpicture}
\caption{Training curves for the best MSE run.}
\end{figure}

\textbf{Key metrics (MSE):}
\begin{itemize}
    \item best eval mean reward: 0.6953 at step 70,000;
    \item final eval mean reward: 0.6537 at step 75,600.
\end{itemize}
This meets the homework success threshold (\(\ge 0.5\)).

\section{Flow Matching Policy Results}
\subsection{Training Curves}
\begin{figure}[H]
\centering
\begin{tikzpicture}
\begin{axis}[
    width=0.48\textwidth,
    height=0.32\textwidth,
    xlabel={Training Step},
    ylabel={Train Loss},
    grid=both,
    title={Flow: Loss vs Step},
]
\addplot[color=blue, line width=1pt] table[x=step, y=loss, col sep=comma] {data/flow_loss.csv};
\end{axis}
\end{tikzpicture}
\hfill
\begin{tikzpicture}
\begin{axis}[
    width=0.48\textwidth,
    height=0.32\textwidth,
    xlabel={Training Step},
    ylabel={Eval Mean Reward},
    ymin=0,
    ymax=1,
    grid=both,
    title={Flow: Reward vs Step},
]
\addplot[color=red, mark=*, line width=1pt] table[x=step, y=reward, col sep=comma] {data/flow_reward.csv};
\end{axis}
\end{tikzpicture}
\caption{Training curves for the best flow matching run.}
\end{figure}

\textbf{Key metrics (Flow):}
\begin{itemize}
    \item best eval mean reward: 0.8865 at step 75,600;
    \item final eval mean reward: 0.8865 at step 75,600.
\end{itemize}
This exceeds the homework success threshold (\(\ge 0.7\)).

\section{Qualitative Comparison (from Evaluation Videos)}
\begin{figure}[H]
\centering
\includegraphics[width=0.95\textwidth]{figures/mse_video_strip.png}
\caption{MSE rollout (episode 3): early, middle, and late frames.}
\end{figure}

\begin{figure}[H]
\centering
\includegraphics[width=0.95\textwidth]{figures/flow_video_strip.png}
\caption{Flow rollout (episode 3): early, middle, and late frames.}
\end{figure}

\textbf{Behavioral observations:}
\begin{itemize}
    \item The MSE policy generally solves many cases but is less consistent at the end of trajectory execution.
    \item The flow policy produces smoother and more reliable progress toward goal completion across the trajectory.
    \item Consistent with these observations, flow achieves substantially higher final and peak rewards than MSE.
\end{itemize}

\section{Conclusion}
Both policies were implemented and trained successfully. The MSE policy reaches solid performance above the required threshold, while flow matching provides a clear improvement in both quantitative reward and qualitative rollout reliability.

\end{document}
